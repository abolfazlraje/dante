%@(#)$Id: bandwidth.tex,v 1.9 2001/11/13 09:17:36 michaels Exp $
%/*
% * Copyright (c) 2001
% *      Inferno Nettverk A/S, Norway.  All rights reserved.
% *
% * Redistribution and use in source and binary forms, with or without
% * modification, are permitted provided that the following conditions
% * are met:
% * 1. The above copyright notice, this list of conditions and the following
% *    disclaimer must appear in all copies of the software, derivative works
% *    or modified versions, and any portions thereof, aswell as in all
% *    supporting documentation.
% * 2. All advertising materials mentioning features or use of this software
% *    must display the following acknowledgement:
% *      This product includes software developed by
% *      Inferno Nettverk A/S, Norway.
% * 3. The name of the author may not be used to endorse or promote products
% *    derived from this software without specific prior written permission.
% *
% * THIS SOFTWARE IS PROVIDED BY THE AUTHOR ``AS IS'' AND ANY EXPRESS OR
% * IMPLIED WARRANTIES, INCLUDING, BUT NOT LIMITED TO, THE IMPLIED WARRANTIES
% * OF MERCHANTABILITY AND FITNESS FOR A PARTICULAR PURPOSE ARE DISCLAIMED.
% * IN NO EVENT SHALL THE AUTHOR BE LIABLE FOR ANY DIRECT, INDIRECT,
% * INCIDENTAL, SPECIAL, EXEMPLARY, OR CONSEQUENTIAL DAMAGES (INCLUDING, BUT
% * NOT LIMITED TO, PROCUREMENT OF SUBSTITUTE GOODS OR SERVICES; LOSS OF USE,
% * DATA, OR PROFITS; OR BUSINESS INTERRUPTION) HOWEVER CAUSED AND ON ANY
% * THEORY OF LIABILITY, WHETHER IN CONTRACT, STRICT LIABILITY, OR TORT
% * (INCLUDING NEGLIGENCE OR OTHERWISE) ARISING IN ANY WAY OUT OF THE USE OF
% * THIS SOFTWARE, EVEN IF ADVISED OF THE POSSIBILITY OF SUCH DAMAGE.
% *
% * Inferno Nettverk A/S requests users of this software to return to
% *
% *  Software Distribution Coordinator  or  sdc@inet.no
% *  Inferno Nettverk A/S
% *  Oslo Research Park
% *  Gaustadall�en 21
% *  NO-0349 Oslo
% *  Norway
% *
% * any improvements or extensions that they make and grant Inferno Nettverk A/S
% * the rights to redistribute these changes.
% *
% */

\documentclass[a4paper, final, twoside, english]{article}
\usepackage[latin1]{inputenc}
\usepackage[T1]{fontenc}
\usepackage{babel}
\usepackage{html}

\title{\emph{Dante}, Module \emph{Bandwidth}}
\author{
        Inferno Nettverk A/S \\
        Oslo Research Park   \\
        Gaustadall�en 21   	\\
        NO-0349 Oslo          \\
        Norway}
\date{$$$$Date: 2001/11/13 09:17:36 $$$$}

\makeindex

\begin{document}
\maketitle
\thispagestyle{empty}

\clearpage
\setcounter{page}{1}

\section{Description}
 The \emph{Bandwidth} module gives you control over how much
 bandwidth the \emph{Dante} server uses on behalf of the 
 clients.

 It can be used to limit bandwidth to non-work related web/ftp
 sites, or to prevent ftp-related traffic from impacting too much
 on interactive telnet/ssh traffic.
 
 It can also be used to give more bandwidth to certain clients
 or for traffic to certain cites.

 In addition, when using the \emph{Dante} \emph{bind extension}, it
 can be used to provide bandwidth control to networkservers
 (like e.g. webservers) that do not support bandwidth control internally.

\section{Syntax}
 The syntax of the \texttt{bandwidth} statement is as follows:

 \verb"bandwidth: <bytes>"

 \texttt{bytes} is the maximum bandwidth, measured in bytes, to use per second.

\section{Semantics}
 The \texttt{bandwidth} statement integrates as a part of
 socks-rules.

 The maximal \texttt{bandwidth} set for a rule will be shared
 by all clients matching that rule.  The \emph{Dante} server
 will distribute the bandwidth to the matching clients in a
 least-recently used fashion, trying to let all clients get a
 fair share.

\section{Examples}
 This section shows several examples of how one could use the
 \emph{bandwidth} module.

 \subsection{Limiting web/http bandwidth}
  The below rule shows how one can limit the bandwidth used
  for webtraffic from the clients on the 10.0.0.0/24 net
  to a total of 10240 bytes, or 100 KiloBytes.

  \begin{verbatim}
pass {
   from: 10.0.0.0/24 to: 0.0.0.0/0 port = http
   command: connect
   bandwidth: 102400
}
  \end{verbatim}

 \subsection{Increasing web/http bandwidth}
  The next rule, if placed before other bandwith-limiting rules,
  shows how one can increase the bandwidth used for webtraffic from
  the clients on the 10.0.0.0/24 net to certain sites.

  In this case, the clients will be able to use 1024000 bytes,
  or one MegaByte, per second when going to the address work.example.com.

  \begin{verbatim}
pass {
   from: 10.0.0.0/24 to: work.example.com port = http
   command: connect
   bandwidth: 1024000
}
  \end{verbatim}

 \subsection{Limiting ftp bandwidth}
  The next rule shows how one can limit the bandwidth used
  for ftp-data for the clients on the 10.0.0.0/24 net
  to a total of 10240 bytes, or 10 kB/s

  This only works for \emph{active} ftp, since for \emph{passive} ftp
  we don't have fixed portnumbers.

  \begin{verbatim}
pass {
   from: 0.0.0.0/0 port = ftp-data to: 10.0.0.0/24
   command: bindreply
   bandwidth: 10240
}
  \end{verbatim}

 \subsection{Limiting bandwidth provided by internal servers to the outside} 
  The next rule shows how one could use the \emph{Dante bind extension}
  together with the \emph{Bandwidth} module to limit the amount of
  data provided by a internal server, in this case, a webserver called 
  \emph{our-webserver.example.com}, to a total of 10240 bytes, or 10 kB/s.

  This requires the webserver to be socksified and the \emph{bind extension}
  to be enabled on both the socksified client and on the \emph{Dante}
  server.

  \begin{verbatim}
pass {
   from: 0.0.0.0/0 to: our-webserver.example.com port = http
   command: bindreply
   bandwidth: 10240
}
  \end{verbatim}



\end{document}
